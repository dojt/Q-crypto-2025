\documentclass[a4paper,10pt,reqno,nonamelimits]{article}
%\usepackage[pagebackref,colorlinks=true,urlcolor=blue,linkcolor=blue,citecolor=blue]{hyperref}
\usepackage[colorlinks=true,urlcolor=blue,linkcolor=blue,citecolor=blue]{hyperref}
%\usepackage[colorlinks=false]{hyperref}

%\usepackage[top=2cm,bottom=2cm,left=1.5cm,right=7cm,marginparwidth=6.5cm]{geometry}
\usepackage{a4wide}

%% \usepackage{makeidx}
%%\usepackage{theorem}
%%\usepackage{array}

%\usepackage{mathptmx}        % PSNFSS
%\usepackage{bookman}         % PSNFSS (wide!)
\usepackage{newcent}         % PSNFSS

\usepackage{amsmath}
\usepackage{amssymb}
\usepackage{amsfonts}
\usepackage{amscd}
\usepackage{amsthm}
\usepackage{amsxtra}
%\usepackage{stmaryrd}
%\usepackage{mathabx}
%\usepackage{eucal}
\usepackage{mathrsfs}
\usepackage{nicefrac}\newcommand{\nfrac}{\nicefrac}
\usepackage{bbold}
%%\usepackage{bm}
\usepackage{graphicx}%\graphicspath{ {image/} }
\usepackage{xcolor}
% \usepackage[raggedright,IT,hang]{subfigure}
%\usepackage{url}
\usepackage{tikz}  \usetikzlibrary{quantikz}
%\usepackage{wrapfig}
\usepackage{enumitem}
% \setlist[enumerate]{leftmargin=0em,itemindent=0em, labelindent=0pt,labelwidth=1.5em,labelsep=.5em, align=left}
% \newlist{txtenum}{enumerate}{1}
% \setlist[txtenum]{leftmargin=0em,itemindent=0em, labelindent=0pt,labelwidth=1em,labelsep=2ex, align=left}
% \begin{enumerate}[resume,label=(\alph*)]...
% \begin{enumerate}[label=\arabic*., leftmargin=1em,itemindent=.5em, labelindent=0pt,labelwidth=1em,labelsep=.5em, align=left]
% * H spacing: leftmargin+itemindent = labelindent+labelwidth+labelsep
% * Sub-numbers: just take ``label*'' instead of ``label''

\usepackage[norelsize,boxed,noend,linesnumbered]{algorithm2e}\DontPrintSemicolon
%\SetKwFor{RepeatTimes}{repeat}{times}{endrepeat}
%\RestyleAlgo{ruled}
% 'norelsize' takes care of incompatibility with amsart in old version of algorithm2e.
%             Remove it once the new version is on the system
% \begin{algorithm}[htbp OR H]
%   Commands: \TitleOfAlgo{}
\SetKwInOut{Input}{Input}\SetKwInOut{Output}{Output}\SetKwInOut{Oracle}{Oracle}
% \For{}{} \Foreach{}{} \If{}{} \eIf{}{}{} \EsleIf{}{} \While{}{} \Return{}

\usepackage{datetime}

\newcounter{mysaveenumi}
% U S A G E :
% \setcounter{mysaveenumi}{\theenumi}
% \begin{enumerate}[(a)]\setcounter{enumi}{\themysaveenumi}
%   ...

%\newtheorem{fact}[theorem]{Fact}
% \numberwithin{theorem}{section}

% \renewcommand{\thesubsection}{\thesection.\alph{subsection}}

% \newcommand{\myparagraph}[1]{\mypar #1}
% \newcommand{\myparagraphwskip}[1]{\smallskip\paragraph{#1}}
% \newcommand{\claimpf}[1]{\myparagraph{\textit{#1}}}
% %


%%%%%%%%%%%%%%%%%%%%%%%%%%%%%%%%%%%%%%%%%%%%%%%%%%%%%%%%%%%%%%%%%%%%%%%%%%%%%%%%%%%%%%%%%%%%%%%%%%%%

\newcommand{\mypar}{\par\medskip\noindent}
\newcommand{\myparbig}{\par\bigskip\noindent}

\newcommand{\rip}[2]{\langle #1 \mid #2 \rangle}
\newcommand{\cip}[2]{( #1 \mid #2 )}

\newcommand{\Nm}[1]{\left\| #1 \right\|} % big norm
\newcommand{\nm}[1]{\lVert #1 \rVert}    % small norm

\DeclareMathOperator{\Span}{Span}
\DeclareMathOperator{\Ker}{Ker}
\DeclareMathOperator{\Img}{Img}
\DeclareMathOperator{\tr}{tr}

\DeclareMathOperator{\re}{re}
\DeclareMathOperator{\im}{im}


\newcommand{\RR}{\mathbb R}
\newcommand{\CC}{\mathbb C}
\newcommand{\KK}{\mathbb K}

\renewcommand{\AA}{\mathbb A}
\newcommand{\BB}{\mathbb B}
\newcommand{\DD}{\mathbb D}
\newcommand{\HH}{\mathbb H}

\newcommand{\NN}{\mathbb N}
\newcommand{\ZZ}{\mathbb Z}

\newcommand{\LL}{\mathbb L}
\newcommand{\MM}{\mathbb M}

\newcommand{\PP}{\mathbb P}

\newcommand{\cA}{\mathcal A}
\newcommand{\cB}{\mathcal B}
\newcommand{\cC}{\mathcal C}
\newcommand{\cH}{\mathcal H}
\newcommand{\cG}{\mathcal G}

\newcommand{\One}{\mathbb{1}}
\newcommand{\one}{\mathbf{1}}

\newcommand{\zero}{\mathbf 0}

\newcommand{\orthogonal}{{\mathrel\bot}}


\newcommand{\ANSWER}[1]{\marginpar{#1}}


\newcommand{\HWNR}{1}
\newcommand{\HandOut}{Tue Feb.~25}
\newcommand{\DueDate}{Tue March~4, 10:00}

%%%%%%%%%%%%%%%%%%%%%%%%%%%%%%%%%%%%%%%%%%%%%%%%%%%%%%%%%%%%%%%%%%%%%%%%%%%%%%%%%%%%%%%%%%%%%%%%%%%%
\begin{document}
\title{\large MTAT.07.024 Quantum Crypto\\[2ex]
  {\tiny
    Assoc.~Prof.~Dirk Oliver Theis\\[-1ex]
    Shahla Novruzova
  }\\[2ex]
  \LARGE Homework \# \HWNR}
%%%%%%%%%%%%%%%%%%%%%%%%%%%%%%%%%%%%%%%%%%%%%%%%%%%%%%%%%%%%%%%%%%%%%%%%%%%%%%%%%%%%%%%%%%%%%%%%%%%%%%%%%%%%%%%%%%%%%%%%%%%%%%%%%%%%
%%
%%
\date{%
  \begin{tabular}{rl}
    Handed out:                 &\HandOut                                           \\[1ex]
    Due:                        &\DueDate                                           \\
    {\small As PDF by email to} &{\small\texttt{shahla.novruzova@ut.ee}}            \\
    {\small\quad subject:}      &{\small\texttt{QCRY-HW\HWNR-}\textit{lastname}}
  \end{tabular}
}
\maketitle

\thispagestyle{empty}

\section{Warm-up: Bell states (20 pts)}
Verify that the following four 2-qubit states form an ONB:
\begin{itemize}
\item $( \ket{00}+\ket{11} )/\sqrt2$
\item $( \ket{00}-\ket{11} )/\sqrt2$
\item $( \ket{01}+\ket{10} )/\sqrt2$
\item $( \ket{01}-\ket{10} )/\sqrt2$
\end{itemize}

\begin{quotation}\color{blue}
  (Your calculations here.)
\end{quotation}

\section{Warm-up: Exponential of Hermitian unitaries (20 pts)}
Let $A$ be a Hermitian (i.e., $A^\dag=A$) unitary (i.e., $A^\dag=A^{-1}$) operator.  Prove that,
for all $\theta\in\RR$,
\begin{equation}
  \exp(i\theta A) = \cos\theta \cdot \One + i\sin\theta\cdot A.
\end{equation}

Recall:
\begin{itemize}
\item $\exp(X) = \sum_{k=0}^\infty \frac{X^k}{k!}$
\item $\cos(X) = \sum_{j=0}^\infty \frac{(-1)^j}{(2j)!} X^{2j}$
\item $\sin(X) = \sum_{j=0}^\infty \frac{(-1)^j}{(2j+1)!} X^{2j+1}$
\end{itemize}

\begin{quotation}\color{blue}
  (Your calculation here.)
\end{quotation}

\section{Efficient communication using entanglement (35 pts)}
Consider the quantum communication protocol in Fig.~\ref{fig:sdc}.

\begin{figure}[htp]
  \centering
  \colorbox{lightgray}{%
    \begin{minipage}{0.9\linewidth}
      \begin{enumerate}[start=0,label=\arabic*.]
      \item Before starting, Alice and Bob share a pair of qubits (i.e., each of them has one qubit) in the state
        \begin{equation*}
          \bigl( \ket0\otimes\ket0+\ket1\otimes\ket1 \bigr)/\sqrt2.
        \end{equation*}

      \item Alice is given 2 bits $x,z \in\{0,1\}$ with the task to send them to Bob.
      \item Alice does the following: If $x=1$, apply an \texttt{X}-gate  to her qubit.
      \item Alice does the following: If $z=1$, apply a  \texttt{Z}-gate  to her qubit.
      \item Alice sends her qubit to Bob.
      \item Bob now has both qubits.
      \item Bob applies a \texttt{CNOT} gate with control on the qubit he received from Alice and target
        on his own qubit.
      \item Bob applies a Hadamard basis-change gate to Alice's qubit.
      \item Bob measures both qubits in the computational basis, and sets
        \begin{itemize}
        \item $z' :=$ outcome from Alice's qubit
        \item $x' :=$ outcome from his own qubit.
        \end{itemize}
      \end{enumerate}
    \end{minipage}%
  }
  \caption{Efficient communication protocol with prior entanglement: 2 bits = 1 qubit + entangled
    resource}\label{fig:sdc}
\end{figure}
\medskip%
Recall that
\begin{itemize}
\item $X = \ket+\bra+ - \ket-\bra-$ and $Z=\ket0\bra0-\ket1\bra1$;
\item The unitary for the Hadamard basis change gate is $\ket0\bra+ + \ket1\bra- = \ket+\bra0 + \ket-\bra1$;
\item ``Apply $U$ to the left-most qubit'' means, apply $U\otimes \One$ to the combined system;
\item The unitary operator for \texttt{CNOT} with control on the left qubit and target on the right qubit is:
  \begin{equation*}
    \ket0\bra0\otimes\One + \ket1\bra1\otimes X.
  \end{equation*}
\end{itemize}


Verify that the protocol is correct: For every choice of $x,z$, with probability~1, Bob ends up with $x'=x$ and
$z'=z$.

\begin{quotation}\color{blue}
  (Your solution here.)
\end{quotation}

\section{No-cloning theorem (25 pts)}
Suppose Alice has a qubit in an unknown state.  She wants to send the qubit to Bob, but also keep a copy for
herself.  Let's say that a ``cloning operator'' is a mapping $E\colon\CC^2\to\CC^2\otimes\CC^2$ with the property
\begin{equation}\label{eq:cloneing-operator}
  E\psi = \psi\otimes\psi.
\end{equation}

Alice wants a cloning operator --- but not being Elon Musk, she's subject the rules of quantum mechanics, so
her~$E$ must be a \emph{linear} operator.

Prove that linear cloning operators don't exist.

\begin{quotation}\color{blue}
  (Your solution here.)
\end{quotation}

\end{document}
%%% Local Variables:
%%% mode: latex
%%% fill-column: 115
%%% End:
